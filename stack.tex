%!Tex Program = xelatex
%\documentclass[a4paper]{article}
\documentclass[a4paper]{ctexart}
\usepackage{xltxtra}
%\setmainfont[Mapping=tex-text]{AR PL UMing CN:style=Light}
%\setmainfont[Mapping=tex-text]{AR PL UKai CN:style=Book}
%\setmainfont[Mapping=tex-text]{WenQuanYi Zen Hei:style=Regular}
%\setmainfont[Mapping=tex-text]{WenQuanYi Zen Hei Sharp:style=Regular}
%\setmainfont[Mapping=tex-text]{AR PL KaitiM GB:style=Regular} 
%\setmainfont[Mapping=tex-text]{AR PL SungtiL GB:style=Regular} 
%\setmainfont[Mapping=tex-text]{WenQuanYi Zen Hei Mono:style=Regular} 

\usepackage{listings}
\usepackage{xcolor}
\usepackage{graphicx}
\usepackage{amsmath}
\usepackage{amsthm}
\usepackage{amssymb}
\usepackage{mathrsfs}
\definecolor{codegreen}{rgb}{0,0.6,0}
\definecolor{codegray}{rgb}{0.5,0.5,0.5}
\definecolor{codepurple}{rgb}{0.58,0,0.82}
\definecolor{backcolour}{rgb}{0.95,0.95,0.92}

\renewcommand{\thefigure}{\arabic{section}.\arabic{figure}}

\lstdefinestyle{mystyle}{
    backgroundcolor=\color{backcolour},   
    commentstyle=\color{codegreen},
    keywordstyle=\color{magenta},
    numberstyle=\tiny\color{codegray},
    stringstyle=\color{codepurple},
    basicstyle=\ttfamily\footnotesize,
    breakatwhitespace=false,         
    breaklines=true,                 
    captionpos=b,                    
    keepspaces=true,                 
    numbers=left,                    
    numbersep=5pt,                  
    showspaces=false,                
    showstringspaces=false,
    showtabs=false,                  
    tabsize=2
}

\lstset{style=mystyle}

\newcommand{\hl}[1]
{\noindent {\bf {#1}}}

% 创建定理环境  
\newtheorem{theorem}{定理}  

\theoremstyle{definition}  
\newtheorem{definition}{定义}[section]

\theoremstyle{definition}  
\newtheorem{example}{例}[section]  


\title{数据结构和算法}
\author{王何宇}
\date{}
\begin{document}
\maketitle
\pagestyle{empty}

\section{栈抽象数据类型(Stack ADT)}

\noindent
栈(stack)是一种线性表,但对插入与删除的位置加以限制:这些操作只能在表的一端进行,这一端称为 \emph{栈顶}(top)。

\subsection{栈模型}

\noindent
栈上的基本操作是 \texttt{push}(等价于插入)与 \texttt{pop}(删除最近插入的元素)。最近插入的元素可以在执行 \texttt{pop} 之前,通过 \texttt{top} 这一操作进行查看。\texttt{pop} 或 \texttt{top} 作用在空栈上,一般被认为是对栈 ADT 的错误操作。另一方面,在执行 \texttt{push} 时若空间耗尽,这是实现层面的限制,而不是 ADT 的错误。

\noindent
栈有时也被称为 \emph{后进先出}(LIFO,last in, first out)表。图 3.23 所示的模型仅表明:\texttt{push} 是向栈输入的操作,而 \texttt{pop} 与 \texttt{top} 是输出操作。常见的附加操作还包括:将栈清空、检测栈是否为空等;但本质上,对栈所能进行的核心操作就是 \texttt{push} 与 \texttt{pop}。

\noindent
图 3.24 展示了在若干次操作之后的一个抽象栈。一般的模型是:总有某个元素位于栈顶,并且这是唯一可见的元素。

\begin{figure}[h]
  \centering
  % 概念性占位图说明
  \fbox{\parbox{0.8\linewidth}{
  栈模型:向栈的输入通过 \texttt{push} 完成;输出通过 \texttt{pop} 与 \texttt{top} 完成。
  }}
  \caption{图 3.23\quad 栈模型:输入由 \texttt{push} 完成;输出由 \texttt{pop} 与 \texttt{top} 完成}
\end{figure}

\begin{figure}[h]
  \centering
  % 概念性占位图说明
  \fbox{\parbox{0.6\linewidth}{
  栈模型:只有位于栈顶的元素可被直接访问(示例堆叠:顶部为 2,其下依次为 4、1、3、6)。
  }}
  \caption{图 3.24\quad 栈模型:只有栈顶元素可被访问}
\end{figure}

\subsection{栈的实现}

\noindent
由于栈本质上是受限的线性表,任何线性表的实现都可以用来实现栈。显然,\texttt{list} 与 \texttt{vector} 都支持栈操作;在 99\% 的情况下,它们是最合理的选择。偶尔在特殊场景中,设计一份专用实现会更快。然而,由于栈操作都是常数时间操作,除非在非常特殊的环境下,这样做通常难以带来可观的改进。

\noindent
针对这些特殊场景,下面给出两种常见的栈实现:一种用链式结构,另一种用数组。两者都可以简化 \texttt{vector} 与 \texttt{list} 中的逻辑,因此不再另外给出代码。

\paragraph{链表实现的栈}
\noindent
第一种实现使用单链表。我们通过在表头插入来完成一次 \texttt{push};通过删除表头元素来完成一次 \texttt{pop}。操作 \texttt{top} 仅查看表头元素并返回其值。有时会把 \texttt{pop} 与 \texttt{top} 合并为一个操作。

\paragraph{数组实现的栈}
\noindent
另一种实现避免了指针链接,且可能是更常见的做法。它复用 \texttt{vector} 的 \texttt{back}、\texttt{push\_back} 与 \texttt{pop\_back},因此实现十分直接。对每个栈,我们维护 \texttt{theArray} 与 \texttt{topOfStack},其中空栈时 \texttt{topOfStack = -1}(这也是空栈的初始化方式)。向栈中 \texttt{push} 某个元素 \texttt{x} 时,先递增 \texttt{topOfStack},再令 \texttt{theArray[topOfStack] = x}。执行 \texttt{pop} 时,返回 \texttt{theArray[topOfStack]} 的值,然后将 \texttt{topOfStack} 递减。

\noindent
请注意,这些操作不仅是常数时间,而且在许多机器上还是非常快的常数时间。在某些体系结构上,整数的 \texttt{push}/\texttt{pop} 可由一条机器指令完成,利用带自动增减地址的寄存器操作。许多现代机器把栈操作作为指令集的一部分,这进一步强化了“栈是仅次于数组、最基础的数据结构”这一观点。

\subsection{应用}

\noindent
并不令人意外的是:一旦限制了线性表可用的操作,这些操作就能被非常快速地实现。真正令人惊讶的是:尽管操作数量不多,它们却非常强大而重要。我们将给出栈的三个典型应用;第三个应用还能深入揭示程序是如何组织的。

\paragraph{符号配对}
\noindent
编译器会检查程序中的语法错误,但经常只是缺少某个符号(比如缺少右花括号或注释起始符)就会让编译器打印出成百上千行诊断信息,却没有指出真正的错误。一个有用的工具是:编写程序检查所有符号是否成对匹配。因此,每一个右花括号、右方括号与右圆括号,都必须与其对应的左符号相匹配。
\bibliography{../bibtexlib/crazyfish.bib}
\bibliographystyle{plain}

\end{document}
